\documentclass[12pt,letterpaper]{article}
\usepackage[spanish,mexico]{babel}
\usepackage[margin=2.5cm]{geometry}
\usepackage{graphicx,wrapfig}
\usepackage{tabularx}
\usepackage{caption}
\usepackage{float}
\usepackage{fancyhdr}
\usepackage{cite}
\usepackage[colorlinks=true,urlcolor=red,citecolor=red,linkcolor=blue,linktocpage,pdfpagelabels,bookmarksnumbered,bookmarksopen]{hyperref}
\usepackage{fontawesome}
\usepackage{booktabs}
\usepackage{multirow}

\pagestyle{fancy}
\rfoot{Abril, 2025}
\cfoot{\thepage}
\lhead{Gestión-hídrica-subcuenca-RH12Ec}
\lfoot{}
\rhead{}
\renewcommand{\footrulewidth}{\headrulewidth}
\setlength{\headheight}{15pt}
\setlength{\parindent}{0cm}
\setlength{\parskip}{0cm}
%\renewcommand{\baselinestretch}{2}
\usepackage{setspace}




%---------------------------------

\begin{document}


\thispagestyle{empty}
\titlepage{


\begin{center}

\Huge{
\textbf{Análisis Hidrológico y Priorización de Gestión Hídrica – Subcuenca RH12Ec}
}
    
\end{center}

\vspace{6cm}

\begin{flushright}

\Huge{Por\\

Eli Kenneth Aguilar-Anzures}
    
\end{flushright}

\vfill

\Huge{Abril, 2025}
}

%%%%%%%%%%%%%%%%%%%%%%%%%

\newpage

\section{Resumen}

En esta prueba técnica se llevó a cabo un análisis del comportamiento hídrico, así como del contexto ambiental y social, en un área de 50 km$^2$ ubicada en la subcuenca RH12Ec, en Jalisco. Se realizaron estudios cartográficos que incluyeron factores como la topografía, la edafología, el uso de suelo y la red de drenaje, lo que permitió identificar siete zonas potenciales para intervenciones de gestión hídrica con capacidad para retener escurrimientos. Además, se analizó la precipitación, el escurrimiento y la infiltración históricos, identificando patrones significativos y elaborando pronósticos para mejorar la gestión del recurso hídrico en la región.

\section{Área de estudio}

La región de interés abarca un cuadrante de 50 km$^2$, ubicado al noreste de la subcuenca hidrográfica RH12Ec, en la porción centro-norte del estado de Jalisco. El área de estudio (AE) se localiza dentro de la zona UTM 13N, entre las coordenadas 662,625.57 m E y 669,695.57 m E (ver figura \ref{fig:loc}).

\begin{figure}[h!]
    \centering
    \includegraphics[width=0.9\linewidth]{figs/mapa localización.png}
    \caption{Mapa de localización del cuadrante utilizado para el análisis.}
    \label{fig:loc}
\end{figure}

El AE comprende porciones de los municipios de San Cristóbal de la Barranca (al norte), Zapopan (al sur) e Ixtlahuacán del Río (al sureste). En su interior se identifican 14 localidades, de las cuales únicamente 8 cuentan con población registrada. La localidad más habitada es Huaxtla, situada en la porción sudoriental del área, con un total de 97 habitantes \cite{6}.\\

La zona de estudio presenta un clima cálido subhúmedo \cite{1,2}, y se ve influenciada por la presencia del río Santiago, el cual fluye en dirección SE-NW, al este del área. Este río se origina en el lago de Chapala, ubicado a unos 75 km al sur, y constituye la principal fuente de agua superficial de la región seleccionada.\\

Actualmente, la región enfrenta serios problemas de sobreexplotación hídrica. El acuífero correspondiente al AE está catalogado como sin disponibilidad y sobreexplotado \cite{1}. Por este motivo, resulta indispensable implementar estrategias de gestión del agua, como la construcción de obras de retención de escurrimientos, que favorezcan la recarga del acuífero y contribuyan al manejo sostenible del recurso hídrico.

\section{Clasificación de uso del suelo y análisis satelital}

La clasificación supervisada del uso de suelo en el AE indica que predomina la vegetación natural de densidad moderada, que ocupa el 63\% del área (ver tabla \ref{tab:t1}).

\begin{figure}[H]
    \centering
    \includegraphics[width=0.9\linewidth]{figs/uso suelo.png}
    \caption{Mapa que muestra la clasificación de uso del suelo en el AE a partir del análisis de una imagen del satélite Landsat 7 con la herramienta dzetsaka \cite{7}.}
    \label{fig:uso}
\end{figure}

\begin{table}[H]
    \centering
    \begin{tabular}{c|c|c}
\hline
Categoría & Superficie (ha) & Porcentaje  \\ \hline
Vegetación densa & 324.27 & 7\% \\
Vegetación moderada & 3134.7 & 63\% \\
Área urbana y suelo expuesto\cite{4} & 1192.05 & 24\% \\
Agricultura & 263.52 & 5\% \\
Cuerpos de agua & 34.56 & 1\% \\ \hline
    \end{tabular}
    \caption{Clasificación de uso de suelo en el AE.}
    \label{tab:t1}
\end{table}



\section{Análisis para la identificación de zonas con potencial para retención de escurrimientos}

Se identificaron siete zonas prioritarias a partir del análisis de elevación, pendiente, edafología, red de drenaje y uso de suelo (ver figura \ref{fig:zon}), siguiendo los siguientes criterios: 

\begin{enumerate}
    \item \textbf{Topografía:} Se analizaron los datos de elevación y pendiente para diferenciar zonas de recarga y áreas donde el agua escurre rápidamente sin infiltrarse (ver figuras \ref{anx:a} y \ref{anx:b}).
    \item \textbf{Edafología:} Se priorizaron suelos permeables (textura arenosa y franco-arenosa) que favorecen la infiltración (ver figura \ref{anx:c}).
    \item \textbf{Uso de suelo y vegetación:} Se consideraron áreas con cobertura vegetal moderada o densa para mejorar la retención hídrica y evitar erosión, así como zonas agrícolas y urbanas donde las intervenciones pueden ser también de prevención de riesgos (ver figura \ref{fig:uso}).
    \item \textbf{Red de drenaje:} Se identificaron áreas cercanas a la red de drenaje donde el escurrimiento natural favorece la acumulación de agua(ver figura \ref{anx:d}).
\end{enumerate}
Se priorizaron zonas que cumplen con:
\begin{itemize}
    \item[\faCheck] Proximidad a la red de drenaje (mayor captación de escurrimientos).
    \item[\faCheck] Suelos permeables (favorecen la infiltración y recarga).
    \item[\faCheck] Pendiente $<$ 25° (permite retención y disminuye la erosión).
    \item[\faCheck] Cobertura vegetal moderada (posible complemento con reforestación).
\end{itemize}

\begin{figure}[H]
    \centering
    \includegraphics[width=1.0\linewidth]{figs/zonas priorización.png}
    \caption{Mapa que muestra las zonas identificadas con potencial para intervenciones de gestión hídrica.}
    \label{fig:zon}
\end{figure}

\section{Análisis del comportamiento hídrico en la subcuenca RH12Ec}

Se calculó el volumen por hectárea tanto de infiltración como de escurrimiento en la subcuenca RH12Ec para cada mes, en el periodo comprendido entre 2013 y 2024 (ver tabla \ref{anx:e}). Este análisis permitió estimar el escurrimiento promedio mensual e identificar que, consistentemente, los meses de enero presentan los valores más altos tanto de escurrimiento como de infiltración.\\

Por otro lado, el análisis de la precipitación media mensual histórica para el mismo periodo reveló que los mayores acumulados de lluvia (en milímetros) ocurren entre los meses de mayo y septiembre. En particular, los años 2013, 2016, 2017 y 2021 registraron los índices más altos de precipitación, lo que sugiere una posible recurrencia de este fenómeno cada 3 a 4 años. Con base en esta tendencia, se estima que durante el periodo mayo–septiembre de 2025 podrían registrarse precipitaciones superiores a los 200 mm (ver figura \ref{fig:precip}).

\begin{figure}[H]
    \centering
    \includegraphics[width=0.9\linewidth]{figs/precip.png}
    \caption{Gráfico de ``pequeñas series temporales múltiples'' que muestran la precipitación media mensual para el periodo 2013-2024.}
    \label{fig:precip}
\end{figure}

\section{Conclusiones}

\begin{itemize}
    \item Existe un problema de sobreexplotación hídrica en el acuífero de la zona, clasificado como sin disponibilidad.
    \item El análisis hidrológico muestra que enero presenta los valores más altos de escurrimiento e infiltración, mientras que los mayores acumulados de lluvia se registran de mayo a septiembre.
    \item Existe una recurrencia de eventos de alta precipitación cada 3-4 años, con posibles máximos superiores a 200 mm en 2025.
    \item Los factores ambientales que influyen en los patrones de escurrimiento e infiltración en el AE son la topografía, el tipo de suelo, la cobertura vegetal y la red de drenajes.
    \item El área abarca tres municipios con presencia de 14 localidades, de las cuales sólo 8 tienen población (principalmente Huaxtla, en el municipio de Zapopan).
    \item Muy poca superficie del área está destinada a la agricultura.
    \item Las localidades cercanas a zonas de recarga o retención podrían beneficiarse directamente de intervenciones, facilitando su aceptación y participación.

\end{itemize}


\begin{thebibliography}{100}

\bibitem{1} IIEG (2024). \textit{Diagnóstico municipal del municipio de San Cristóbal de la Barranca}. Disponible en: https://iieg.gob.mx/ns/wp-content/uploads/2024/08/San-Cristobal-de-la-Barranca.pdf

\bibitem{2} IIEG (2024). \textit{Diagnóstico municipal del municipio de Zapopan}. Disponible en: https://iieg.gob.mx/ns/wp-content/uploads/2024/08/Zapopan.pdf

\bibitem{3} INEGI (2007). \textit{Conjunto de Datos Vectorial Edafológico. Escala 1:250 000 Serie II Continuo Nacional Guadalajara}. Disponible en: https://www.inegi.org.mx/app/biblioteca/ficha.html?upc=702825235529

\bibitem{4} INEGI (2010). \textit{Aspectos técnicos de las imágenes Landsat}. Disponible en: https://www.inegi.org.mx/contenidos/temas/imagenes/imgLANDSAT/doc/Aspectos\_\\tecnicos\_landsat.pdf

\bibitem{5} INEGI (2010). \textit{Red hidrográfica. Escala 1:50 000. Edición 2.0. Subcuenca hidrográfica RH12Ec R. Verde-P. Santa Rosa. Cuenca R. Santiago-Guadalajara. RH Lerma-Santiago}. Disponible en: https://www.inegi.org.mx/app/biblioteca/ficha.html?upc=889463131342

\bibitem{6} INEGI (2019). \textit{Conjunto de Datos Vectoriales de Información Topográfica F13D55 Tesistán escala 1:50 000}. Disponible en: https://www.inegi.org.mx/app/biblioteca/ficha.html?upc=889463831440

\bibitem{7} Karasiak, N. (2016). \textit{Dzetsaka Qgis Classification plugin}. DOI: 10.5281/zenodo.2552284

\end{thebibliography}



\newpage

\section*{Anexos}
\begin{figure}[H]
    \caption{Anexo A.}
    \label{anx:a}
    \centering
    \includegraphics[width=0.8\linewidth]{figs/elevación.png}
\end{figure}
\begin{figure}[H]
    \caption{Anexo B.}
    \label{anx:b}
    \centering
    \includegraphics[width=0.8\linewidth]{figs/pendiente.png}
\end{figure}

\newpage

\begin{figure}[H]
    \caption{Anexo C.}
    \label{anx:c}
    \centering
    \includegraphics[width=0.8\linewidth]{figs/edafología.png}
\end{figure}
\begin{figure}[H]
    \caption{Anexo D.}
    \label{anx:d}
    \centering
    \includegraphics[width=0.8\linewidth]{figs/red_uso_suelo.png}
\end{figure}

\newpage

\begin{table}[H]
\caption{ANEXO E. Volumen de infiltración y escurrimiento por hectárea para cada mes de cada año.}
\label{anx:e}
\centering
\begin{tabular}{l|c|c|c}
\hline
Año & Mes & Volumen Infiltración ($m^3$) & Volumen Escurrimiento ($m^3$) \\
\hline
\multirow[t]{12}{*}{2013} & 1 & 2343173.93 & 3119998.93 \\
 & 2 & 191.55 & 245320.01 \\
 & 3 & 0.26 & 313318.49 \\
 & 4 & 156138.70 & 2318877.42 \\
 & 5 & 185256.67 & 4682718.51 \\
 & 6 & 678939.85 & 7968465.43 \\
 & 7 & 12941477.19 & 32902773.14 \\
 & 8 & 10805886.82 & 24741053.92 \\
 & 9 & 18928043.39 & 38840542.55 \\
 & 10 & 4130513.29 & 13240081.78 \\
 & 11 & 683975.23 & 2936276.22 \\
 & 12 & 865442.18 & 2920673.33 \\
\cline{1-4}
\multirow[t]{12}{*}{2014} & 1 & 2343173.93 & 3119998.93 \\
 & 2 & 191.55 & 245320.01 \\
 & 3 & 0.26 & 313318.49 \\
 & 4 & 156138.70 & 2318877.42 \\
 & 5 & 185256.67 & 4682718.51 \\
 & 6 & 678939.85 & 7968465.43 \\
 & 7 & 12941477.19 & 32902773.14 \\
 & 8 & 10805886.82 & 24741053.92 \\
 & 9 & 18928043.39 & 38840542.55 \\
 & 10 & 4130513.29 & 13240081.78 \\
 & 11 & 683975.23 & 2936276.22 \\
 & 12 & 865442.18 & 2920673.33 \\
\cline{1-4}
\multirow[t]{12}{*}{2015} & 1 & 2343173.93 & 3119998.93 \\
 & 2 & 191.55 & 245320.01 \\
 & 3 & 0.26 & 313318.49 \\
 & 4 & 156138.70 & 2318877.42 \\
 & 5 & 185256.67 & 4682718.51 \\
 & 6 & 678939.85 & 7968465.43 \\
 & 7 & 12941477.19 & 32902773.14 \\
 & 8 & 10805886.82 & 24741053.92 \\
 & 9 & 18928043.39 & 38840542.55 \\
 & 10 & 4130513.29 & 13240081.78 \\
 & 11 & 683975.23 & 2936276.22 \\
 & 12 & 865442.18 & 2920673.33 \\
\cline{1-4}
\multirow[t]{12}{*}{2016} & 1 & 2343173.93 & 3119998.93 \\
 & 2 & 191.55 & 245320.01 \\
 & 3 & 0.26 & 313318.49 \\
 & 4 & 156138.70 & 2318877.42 \\
 & 5 & 185256.67 & 4682718.51 \\
\cline{1-4}
\hline
\end{tabular}
\end{table}

\begin{table}[H]
\centering
\begin{tabular}{l|c|c|c}
\hline
Año & Mes & Volumen Infiltración ($m^3$) & Volumen Escurrimiento ($m^3$) \\
\cline{1-4}
\multirow[t]{12}{*}{2016} & 6 & 678939.85 & 7968465.43 \\
& 7 & 12941477.19 & 32902773.14 \\ 
 & 8 & 10805886.82 & 24741053.92 \\
 & 9 & 18928043.39 & 38840542.55 \\
 & 10 & 4130513.29 & 13240081.78 \\
 & 11 & 683975.23 & 2936276.22 \\
 & 12 & 865442.18 & 2920673.33 \\
\cline{1-4}
\multirow[t]{12}{*}{2017} & 1 & 2343173.93 & 3119998.93 \\
 & 2 & 191.55 & 245320.01 \\
 & 3 & 0.26 & 313318.49 \\
 & 4 & 156138.70 & 2318877.42 \\
 & 5 & 185256.67 & 4682718.51 \\
 & 6 & 678939.85 & 7968465.43 \\
 & 7 & 12941477.19 & 32902773.14 \\
 & 8 & 10805886.82 & 24741053.92 \\
 & 9 & 18928043.39 & 38840542.55 \\
 & 10 & 4130513.29 & 13240081.78 \\
 & 11 & 683975.23 & 2936276.22 \\
 & 12 & 865442.18 & 2920673.33 \\
\cline{1-4}
\multirow[t]{12}{*}{2018} & 1 & 2343173.93 & 3119998.93 \\
 & 2 & 191.55 & 245320.01 \\
 & 3 & 0.26 & 313318.49 \\
 & 4 & 156138.70 & 2318877.42 \\
 & 5 & 185256.67 & 4682718.51 \\
 & 6 & 678939.85 & 7968465.43 \\
 & 7 & 12941477.19 & 32902773.14 \\
 & 8 & 10805886.82 & 24741053.92 \\
 & 9 & 18928043.39 & 38840542.55 \\
 & 10 & 4130513.29 & 13240081.78 \\
 & 11 & 683975.23 & 2936276.22 \\
 & 12 & 865442.18 & 2920673.33 \\
\cline{1-4}
\multirow[t]{12}{*}{2019} & 1 & 2343173.93 & 3119998.93 \\
 & 2 & 191.55 & 245320.01 \\
 & 3 & 0.26 & 313318.49 \\
 & 4 & 156138.70 & 2318877.42 \\
 & 5 & 185256.67 & 4682718.51 \\
 & 6 & 678939.85 & 7968465.43 \\
 & 7 & 12941477.19 & 32902773.14 \\
 & 8 & 10805886.82 & 24741053.92 \\
 & 9 & 18928043.39 & 38840542.55 \\
 & 10 & 4130513.29 & 13240081.78 \\
 & 11 & 683975.23 & 2936276.22 \\
 & 12 & 865442.18 & 2920673.33 \\
\cline{1-4}
\hline
\end{tabular}
\end{table}

\begin{table}[H]
\centering
\begin{tabular}{l|c|c|c}
\hline
Año & Mes & Volumen Infiltración ($m^3$) & Volumen Escurrimiento ($m^3$) \\
\cline{1-4}
\multirow[t]{12}{*}{2020} & 1 & 2343173.93 & 3119998.93 \\
 & 2 & 191.55 & 245320.01 \\
 & 3 & 0.26 & 313318.49 \\
 & 4 & 156138.70 & 2318877.42 \\
 & 5 & 185256.67 & 4682718.51 \\
 & 6 & 678939.85 & 7968465.43 \\
 & 7 & 12941477.19 & 32902773.14 \\
 & 8 & 10805886.82 & 24741053.92 \\
 & 9 & 18928043.39 & 38840542.55 \\
 & 10 & 4130513.29 & 13240081.78 \\
 & 11 & 683975.23 & 2936276.22 \\
 & 12 & 865442.18 & 2920673.33 \\
\cline{1-4}
\multirow[t]{12}{*}{2021} & 1 & 2343173.93 & 3119998.93 \\
 & 2 & 191.55 & 245320.01 \\
 & 3 & 0.26 & 313318.49 \\
 & 4 & 156138.70 & 2318877.42 \\
 & 5 & 185256.67 & 4682718.51 \\
 & 6 & 678939.85 & 7968465.43 \\
 & 7 & 12941477.19 & 32902773.14 \\
 & 8 & 10805886.82 & 24741053.92 \\
 & 9 & 18928043.39 & 38840542.55 \\
 & 10 & 4130513.29 & 13240081.78 \\
 & 11 & 683975.23 & 2936276.22 \\
 & 12 & 865442.18 & 2920673.33 \\
\cline{1-4}
\multirow[t]{12}{*}{2022} & 1 & 2343173.93 & 3119998.93 \\
 & 2 & 191.55 & 245320.01 \\
 & 3 & 0.26 & 313318.49 \\
 & 4 & 156138.70 & 2318877.42 \\
 & 5 & 185256.67 & 4682718.51 \\
 & 6 & 678939.85 & 7968465.43 \\
 & 7 & 12941477.19 & 32902773.14 \\
 & 8 & 10805886.82 & 24741053.92 \\
 & 9 & 18928043.39 & 38840542.55 \\
 & 10 & 4130513.29 & 13240081.78 \\
 & 11 & 683975.23 & 2936276.22 \\
 & 12 & 865442.18 & 2920673.33 \\
\cline{1-4}
\multirow[t]{12}{*}{2023} & 1 & 2343173.93 & 3119998.93 \\
 & 2 & 191.55 & 245320.01 \\
 & 3 & 0.26 & 313318.49 \\
 & 4 & 156138.70 & 2318877.42 \\
 & 5 & 185256.67 & 4682718.51 \\
 & 6 & 678939.85 & 7968465.43 \\
 & 7 & 12941477.19 & 32902773.14 \\
\cline{1-4}
\hline
\end{tabular}
\end{table}

\newpage
\begin{table}[H]
\centering
\begin{tabular}{l|c|c|c}
\hline
Año & Mes & Volumen Infiltración ($m^3$) & Volumen Escurrimiento ($m^3$) \\
\cline{1-4}
\multirow[t]{12}{*}{2023} & 8 & 10805886.82 & 24741053.92 \\
 & 9 & 18928043.39 & 38840542.55 \\
 & 10 & 4130513.29 & 13240081.78 \\
 & 11 & 683975.23 & 2936276.22 \\
 & 12 & 865442.18 & 2920673.33 \\
\cline{1-4}
\multirow[t]{12}{*}{2024} & 1 & 2343173.93 & 3119998.93 \\
 & 2 & 191.55 & 245320.01 \\
 & 3 & 0.26 & 313318.49 \\
 & 4 & 156138.70 & 2318877.42 \\
 & 5 & 185256.67 & 4682718.51 \\
 & 6 & 678939.85 & 7968465.43 \\
 & 7 & 12941477.19 & 32902773.14 \\
 & 8 & 10805886.82 & 24741053.92 \\
 & 9 & 18928043.39 & 38840542.55 \\
 & 10 & 4130513.29 & 13240081.78 \\
 & 11 & 683975.23 & 2936276.22 \\
 & 12 & 865442.18 & 2920673.33 \\
\cline{1-4}
\hline
\end{tabular}
\end{table}
















\end{document}
